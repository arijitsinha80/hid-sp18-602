% status: 100
% chapter: OpenCV

\title{Cloud AutoML}


\author{Keerthi Naredla}
\affiliation{
  \institution{Indiana University Bloomington}
  \city{Bloomington}
  \state{Indiana}
  \postcode{47404}
  \country{USA}}
\email{knaredla@iu.edu}



% The default list of authors is too long for headers}
\renewcommand{\shortauthors}{Keerthi}



\begin{abstract}
  Cloud AutoML is the state-of -the -art tool to design high-quality
  training and large-scale capable custom ML models. Using this, a
  user with no or less ML knowledge can build ML model in short span
  of time, low-cost hardware and infrastructure. It is a revolution
  by Google team to get ML to small businesses with less AI/ML
  expertises. It is based on 2 important technologies transfer
  learning and neural architecture search technology. The Neural
  Architecture search uses the concept of learning to learn or meta
  learning with an auto-regressive controller which builds a neural
  network on learning from feedback of child network. As it is
  immensly expensive to compute and execute deep neural networks,
  Google Cloud Services such as Google Compute Engine, Google Cloud
  Storage support Cloud AutoML. In additon to this, Google API's
  provide the ability to customize pre-trained models in Cloud AutoML.
  This section gives a brief introduction to Cloud AutoML, the
  technology behind its functioning,and the importance of Google Cloud
  in Cloud AutoML.
\end{abstract}

\keywords{hid-sp18-602,google,cloud,automl,neural,search,network}


\maketitle

\section{Introduction}

Cloud AutoML is an innovative tool with simple graphical user
interface to train and test users custom machine learning models. This
is a result of collaborative effort of Google Cloud AI, Google Brain
and other Google AI teams. The main purpose of developing Cloud AutoML
is to enable users and businesses with limited machine learning
expertise to easily build and train high quality custom ML models. It
is built on Google learning to learn, transfer learning, and Neural
Architecture Search technologies~\cite{hid-sp18-602-cloud-automl-launch}.


Cloud AutoML is a suit of Machine Learning products. Google has
recently launched first product under Cloud AutoML: AutoML Vision
which is a service to access a pre-trained model or create a custom ML
models using Google Cloud Services, for image recognition, detecting
image content, classifying images and image-based recommendation
system. It offers drag-and-drop interface to upload images, train and
manage models. Similarly Google is working to support integration of
it's poweful API’s into Cloud AutoML. Few of the API's that are in
real demand are, Google Cloud Video Intelligence API which makes
videos searchable, Google Text-to-Speech, Speech-to-Text which is
highly used in smart-home devices, automation tasks, Natural Language
API which is useful for text-analysis, extracting useful information
from users, Google Translation which is useful for language detection
,conversion and Google DialogFlow which highly improves interaction
and conversation with voice assistants. Thus using Cloud AutoML, a
business can customize ML model according to their needs by selecting
any one or a combination of these API's
~\cite{hid-sp18-602-cloud-automl-main}.


Just like cloud servers are used by several companies, small and big,
without any knowledge of underlying complexity involved in storing,
distribution and processing, the cloud automl can be used to build
customized neural network that serves the purpose without actually
understanding the complexity of generating a model
~\cite{hid-sp18-602-cloud-automl-impact}.  Not only that, it is
time-efficient and high-accurate because the base model is already
pre-trained on immense data-archives and the resources used by
Google. Also, AutoML generated models run instantly on Cloud ML
infrastructure leveraging the hardware and powering Google’s own
cloud.


With these major advantages, Cloud AutoML Vision is already in use by
many good companies across the globe.  For instance, shopDisney uses
label-detecting feature with Cloud AutoML technology to build vision
models in order to label products with Disney characters, and product
categories.  Also, these annotations are integrated into search engine
for better product recommendations. Urban outfitters use cloud automl
to automate product attribution process to recognize products like
patterns and dressstyles, which is very useful in terms of accurate
search results, product recommendation.  Zoological Society of London
are actively using AutoML Vision to categorize different animals from
the images captured in order to analyze and understand animal motions,
distribution and human impact on
wildlife~\cite{hid-sp18-602-cloud-automl-launch}.

\section{Mechnism}
This is section is based on Google's 2017 paper on Neural Architecture
Search with Reinforcement Learning, by Bareet Zoph,Quoc V.Le Google
Brain Team~\cite{hid-sp18-602-cloud-automl-mechanism}.  Neural
Architecture Search with reinforcement learning is the basis of the
Google Automl. In addition to this, the concept of Transfer learning
which is to make use of pre-trained models, in order to build a custom
model with small changes to base model, is the motivation for Cloud
AutoML~\cite{hid-sp18-602-transfer-learning}.  Many of the previously
proposed methods like Hyperparameter optimization, Neuro-evolution
algorithm, sequence-to-sequence learning lack some of the core
concepts like able to generate variable-length network, able to work
at large-scale, learning from reward signal without any
supervised/manual intervention, making use of previously learnt
information or feedback framework which is also called learning to
learn or meta learning.

A recurrent neural network trained with reinforcement learning
generates a convolutional architecture. In this model, the concept of
reinforcement learning has a RNN called controller which is used to
generate a variable-length string, by constantly training the network
with results of child-network on the validation-set. And the result
which is in fact know as reward signal is processed through policy
gradient-method to further update the controller. With increase in
number of iterations of this process, the neural network grows,
resulting in higher accuracy. The key additions to this neural search
architecture model are: (1) Parameter-server scheme which uses
distributed training of child network and allows asynchronous updates
to the controller. This parallelism speed up the training
process,rather than spending hours on each child network. (2) Skip
connections and branching layers are used by the controller in order
to increase the search space and also the complexity of the
architecture as a whole rather than standard RNN.That is with skip
connections the controller can decide on what input layers should link
to the current layer, rather than choosing just the previous layer.


Using Neural Architecture Search the novel model built on CIFAR-10
dataset is called ConvNet for image-recoginition, has 3.65 test set
error,that is 1.05x faster than best human-invented models and novel
recurrent cell designed for Natural Lanaguage Processing on Penn
Treebank dataset, results in 3.6 perplexity better than any previous
RNN, LSTM models. Usually, to build ML models for such large datasets
take not only enormous amount of time but also immense effort of ML
experts would result in better architecture. But with the concept of
Neural Architecture Search, a machine can generate a recurrent neural
network that is far better than experts built state-of-the art
models. Hence Neural Architecture Search achives building best models
from scratch, with less human intervention, less time and high test
set accuracy~\cite{hid-sp18-602-cloud-automl-mechanism}.
 
\section{Role of Google Cloud}
Google Cloud is the one of prerequisites for functioning of Cloud AI
services such as Cloud ML Engine, Dialogflow,Google Cloud Job Search
and Discovery and other Google API's. Using Cloud Machine Learning
Engine, data scientist and ML expertise can work together to design a
ML model with help of Tensorflow and then train, test the model on
large scale processed data deployed on a cluster
~\cite{hid-sp18-602-cloud-mlengine-training}.


Although Cloud ML Engine supports training of the model to the extent
of high accurate prediction rate, we need Google Cloud Storage for
storing input data for training and testing, staging all the
dependencies of the custom ML model into a trainer package, writing
training artifact and storing training and prediction output
files~\cite{hid-sp18-602-google-cloud-storage}.  Similarly, Customized
Cloud Vision models which are usually deep neural networks have
several millions of nodes, that need to undergo multiple training
cycles to acquire high performance, high accuracy, and this results in
computationally intensive even with special hardware infrastructure
~\cite{hid-sp18-602-cloud-ai}. Fei-Fei-Li, Cheif Scientist Google
Cloud AI, mentioned that ``Google’s infrastructure is the solution to
speed training times. Google has specialized ASIC, GPU and TP hardware
in its cloud to accelerate training and improve the ROI with on-demand
cloud resource utilization''~\cite{hid-sp18-602-cloud-ai}.


Google's Cloud TPU:Tensor Processing Unit is crucial for lowering the
time required to train comutationally intensive models. It is built
with application-specific integrated circuits, and consists of 180
teraflops computing power with 64 gigabytes of high-bandwidth storage
memory~\cite{hid-sp18-602-cloud-tpu-overview}.  It is flexible to
shift models running on Tensorflow to Cloud TPU. And it is important
to consider Cloud TPU, especially if the training dataset is huge,
increasing, and model takes several cycles to achive accurate
prediction, as it leverages requirement of local datacenters setup.
Also with XLA just-in-time compiler and Cloud TPU hardware which has
matrix unit (MXU) it is possible to train large models with very large
batch size that typically takes months and years in few weeks or
months. But Cloud TPU cannot be used if the neural network isn't built
using Tensorflow or the main training loop of tensorflow program
consists of operations, in that case GPU: Graphic Processing Units can
be used instead of TPU. GPU's are also useful to accelerate machine
learning workload and these can be simply added to VM instance on
which model is running~\cite{hid-sp18-602-cloud-tpu}. Therefore, it is
important to note that TPU's are not the only option,rather Google
makes use of GPU, CPU's to run machine learning worklaods on Compute
Engine when required.  Thus, Gloogle Cloud Services comes with all of
these resouces which play a key role in Cloud Automl and other Cloud
AI products.

\section{Conclusion}
Cloud AutoML is certainly a mission to get profound concepts ML,AI,
deep learning neural networks into usage for any company. Google terms
this mission as ``democratizing AI, point-and -click AI for all'', as
it is not just about leveraging the technology to build deep layers of
neural network but it is also about leveraging the the
infrastructure through Googgle Cloud Services especially Cloud TPU,
required for getting the ML model into production in large scale and
accuracy.

Moreover, with powerful Google API's such as NLP, Speech, Vision,
Translation, building customized ML models becomes feasible and
flexible. Thus Google's Cloud AutoML is sucessful in solving the issue
of requiring highly skilled and experianced Machine learning experts
to develop advanced neural networks and significant amount of time
taken to build such models manaully can now be machine-generated with
Neural Architecture Search with Reinforcement Learning and reused with
Transfer Learning. Hence any company can have AI/ML products that
could match the quality and speed of Google AI products.

\begin{acks}

  The authors would like to thank Dr.Gregor von Laszewski for his
  support and suggestions to write this paper.
  

\end{acks}

\bibliographystyle{ACM-Reference-Format}
\bibliography{report}
